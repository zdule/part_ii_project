% The master copy of this demo dissertation is held on my filespace
% on the cl file serve (/homes/mr/teaching/demodissert/)

% Last updated by MR on 2 August 2001

\documentclass[12pt,twoside,notitlepage]{report}

\usepackage{a4}
\usepackage{verbatim}
%\usepackage{navigator}
\usepackage{hyperref}
\hypersetup{
    colorlinks,
    citecolor={black},
    filecolor={black},
    linkcolor={black},
    urlcolor={black}
}

%\input{epsf}                            % to allow postscript inclusions
% On thor and CUS read top of file:
%     /opt/TeX/lib/texmf/tex/dvips/epsf.sty
% On CL machines read:
%     /usr/lib/tex/macros/dvips/epsf.tex



\raggedbottom                           % try to avoid widows and orphans
\sloppy
\clubpenalty1000%
\widowpenalty1000%

\addtolength{\oddsidemargin}{6mm}       % adjust margins
\addtolength{\evensidemargin}{-8mm}

\renewcommand{\baselinestretch}{1.1}    % adjust line spacing to make
                                        % more readable
\usepackage{biblatex}
\addbibresource{refs.bib}
\begin{document}
%%%%%%%%%%%%%%%%%%%%%%%%%%%%%%%%%%%%%%%%%%%%%%%%%%%%%%%%%%%%%%%%%%%%%%%%
% Title


\pagestyle{empty}

\hfill{\LARGE \bf Dušan Živanović}

\vspace*{60mm}
\begin{center}
\Huge
{\bf Optimizing tracing instrumentation inside the Linux kernel} \\
\vspace*{5mm}
Computer Science Tripos -- Part II \\
\vspace*{5mm}
Tirinty College \\
\vspace*{5mm}
2020
\end{center}


%%%%%%%%%%%%%%%%%%%%%%%%%%%%%%%%%%%%%%%%%%%%%%%%%%%%%%%%%%%%%%%%%%%%%%%%%%%%%%
% Proforma, table of contents and list of figures

\setcounter{page}{1}
\pagenumbering{roman}
\pagestyle{plain}

\chapter*{Acknowledgements}
\chapter*{Declaration}
    I, Dušan Živanović of Trinity College, being a candidate for Part II of the Computer Science Tripos, hereby declare that this dissertation and the work described in it are my own work, unaided except as may be specified below, and that the dissertation does not contain material that has already been used to any substantial extent for a comparable purpose. 

    I, Dušan Živanović of Trinity College, am content for my dissertation to be made available to the students and staff of the University. 

    Signed 

    Date 

    \bigskip
    \leftline{Signed [signature]}

    \medskip
    \leftline{Date [date]}


\chapter*{Proforma}


{\large
\begin{tabular}{ll}
Candidate Number:   & \bf TBD                   \\
Project Title:      & \bf Optimizing tracing instrumentation \\
                    & \bf inside the Linux kernel \\
Examination:        & \bf Computer Science Tripos -- Part II, 2020      \\
Word Count:         & \bf TBD \\
Line Count:         & \bf TBD \\
Project Originator: & Dr Lucian Carata and Dr Ripduman Sohan     \\
Supervisor:         & Dr Lucian Carata                  \\ 
\end{tabular}
}


\section*{Original Aims of the Project}


\section*{Work Completed}


\section*{Special Difficulties}



\tableofcontents

\listoffigures

%%%%%%%%%%%%%%%%%%%%%%%%%%%%%%%%%%%%%%%%%%%%%%%%%%%%%%%%%%%%%%%%%%%%%%%
% now for the chapters

\cleardoublepage        % just to make sure before the page numbering
                        % is changed

\setcounter{page}{1}
\pagenumbering{arabic}
\pagestyle{headings}

\chapter{Introduction}
    This dissertation introduces kambpf, a flexible, low-overhead function-call tracing system for the Linux kernel. Compared to the tracing capabilities currently in the Linux kernel, kambpf has a lower overhead and the added flexibility instrumenting any subset of call sites of a given function (and add massive amounts of probes). Compared to the kamprobes project on which it builds upon, kambpf provides the flexibility of being programmable with eBPF, and an interface from the user space. 

    The kambpf system is available as a loadable kernel module with its functionality exported to the user space via a simple to use C library.

    \section{Motivation}

        Dynamic tracing is a way to gain insight into the execution of a program. It is done by inserting additional instructions into to binary of a process at runtime to record data, such as values of function arguments, stack traces or timing. The effect is similar to inserting printf's into the source code, but it can be enabled and disabled at runtime without recompiling or even stoping the program, and with no cost when tracing is disabled.

        Dynamic tracing is especially useful for examining long running processes in production. In such a scenario, restrating the process would be too disruptive. On the other hand replicating the conditions in a test environment might be impossible due to a unique combination of hardware, configuration, other applications running concurrently, and the workload. 

        Useful questions that dynamic tracing can answer are which part of the code, or which external resource is the bottleneck; what values is this function usually called with; why does a piece of code exibit pathological behaviour; how much resources is dedicated to running this process. With answers to these questions we can fix performance problems, optimize the code, choose which resource to buy more of.

        It is particularly convenient to instrument the operating system kernel because it is the meeting point of all processes and the hardware resources. For example, just by instrumenting kernel functions are called we can record when processes are scheduled, how much data they send and to whom, and the latency and throughput a hard drive.

        An assumption present in the implementation available in the Linux kernel (TODO: too indirect) is that the user will want to instrument a few specific function calls related to the issue they are debugging. For this reason the implementation hasn't been designed to support large numbers of probes. A use case that requires large number of probes is measuring how much time is spend executing functions in each file. To achieve this probes would be added to all function calls made from any file into other files. This requires massive number of probes, and is something that current implementations struggle with as they scale poorly with the number of probes used.

        When analysing performance by tracing it is most important that the overhead is as small as possible otherwise our measurements might show that the tracing itself is the bottleneck, or in other way show a result skewed by tracing. As is often the case the easiest way to reduce the cost of tracing is simply to do less of it, which mostly refers to saving less data about events into buffers. This is usually done in two ways. Firstly, if a data point is going to be discarded, then it should be discarded as soon as possible, before writing it to a buffer. Secondly, if we are only interested in a summary of the collected data, such as an average, then the tracing instrumentation should maintain a average while tracing instead of recording all data to a buffer and processing it afterwards. Because the rules for which events should be discarded and how the collected data should be summarized vary between different use cases the tracing system should allow specifying the action taken in response to events.

		Even more overhead can be avoided if the tracing code doesn't even run when no data should be collected. Lets imagine that we to analyse the high memory usage of a device driver. To do so we might want to see where in the code is most memory allocated, and we can collect this data by tracing the function \texttt{kmalloc}. As we are only interested in allocations coming from that particular driver, the handler that we specify to run in response to \texttt{kmalloc} being called will check to see if the function is called from our driver and only if it is will collect the stack trace and the size of memory allocated. For all of the much more common calls to kmalloc happening from outside the driver this whole process will have been unnecessary overhead. This overhead can be eliminated by tracing the just the individual function call instructions of kmalloc that we are interested in, instead of the kmalloc function itself. The practical difference is that the tracing code is dynamically inserted into multiple call sites calling kmalloc instead of inserting the instrumentation at the beginning of the kmalloc function itself. Although this does more work upfront, there is no cost for the calls originating outside of our driver. Because we can better describe what events we are interested in by specifying the individual call sites of a function, tracing has less overhead.

		Finally, specifying the actions to be taken in response to events, the event handlers, should be simple and safe, preferably not requiring custom kernel modules, in which any mistake could crash the kernel. A new addition to the Linux kernel, the eBPF virtual machine, provides a safe way to run user supplied programs. The tracing systems in the Linux kernel all support adding eBPF programs as event handlers, but none of them support call site probes efficiently.

		The goal of this project is thus to create a flexible and efficient tracing system for the Linux kernel, that can trace individual call sites, and can take eBPF programs as event handlers. Furthermore I will measure the overhead of this new tracing mechanism, and compare it to the overhead of the existing solutions. Finally, in order to verify that the tracing system can be used for a real world application I will use it to analyse where the kernel is spending time and waiting when sending a TCP packet. (TBC)


    \section{Previous work}
		\begin{itemize}
			\item \texttt{ftrace} is a Linux kernel function tracer. It can be used to trace any function in the Linux kernel and can take eBPF programs as event handlers. It does not support tracing individual call instructions

			\item \texttt{kprobe}s are another tracing mechanism of the Linux kernel and like \texttt{ftrace} it also support eBPF. Additionally it can instrument any instruction. Unfortunately, tracing call instructions is done by inserting breakpoint instructions, which is inefficient. \texttt{kprobe}s also scale poorly with the number of probes is high, which is the case when call instructions are individually instrumented.

			\item \texttt{kamprobes} a low-overhead tracing mechanism that supports tracing many call instructions. The main limitation of this probing mechanism is that event handlers must be functions in the Linux kernel, so a user that needs custom filtering and aggregation needs to write a loadable kernel module. Furthermore, \texttt{kamprobes} have been developed as a part of the \texttt{resourceful} project and currently don't provide an interface outside of the Linux kernel.
		\end{itemize}

	\section{Summary}

		This chapter introduced dynamic tracing, the need for low-overhead probes, filtering and aggregating events in event handlers. Additionally it demonstrated why instrumenting function call instructions is more flexible than instrumenting functions themselves. Finally the goal of the dissertation was stated as developing a flexible and efficient system for tracing in the Linux kernel, and the Previous work section described why current solutions are unsatisfactory.

		Mention that the plain is to add eBPF support to kamprobes in the introduction?

\chapter{Preparation}
 
\section {Requirements analysis} 
    The main goal of the project was to make new uses of tracing possible with scalable support of large numbers of probes. The design I find should be efficient and reasonably simple to deploy and use. These ideas are expanded into requirements listed bellow. The requirements explain decisions made during the implementation.

    \begin{itemize}
        \item \textbf{Scalable support for large numbers of probes} -- This is the reason other people should care about the project. It makes more information about the system accessible.
        \item \textbf{Precise specification of which function calls to trace} -- The intent is to keep overhead low by tracing only the function calls we need to trace.
        \item \textbf{Performance and flexibility of custom event handler} -- Custom event handlers can be used to avoid recording all information regarding a function call. Instead event handlers can process and record only the needed information.
        \item \textbf{Use without writing kernel code} -- Writing kernel code is time consuming and error-prone. Therefore, not requiring custom kernel code to add probes and for event handlers will make my design a lot easier to use.
        \item \textbf{Support for x86-64}-- The scope of the design is limited to the x86-64 Architecture because it is the most common architecture for servers. Additionally, the code will architecture specific so supporting multiple architectures requires a lot of additional work.
    \end{itemize}
    
    The project idea described in the proposal is to satisfy these requirements by building on the existing implementation for scalable, low-overhead instruction call probes, called kamprobes, and use the eBPF virtual machine to specify event handlers. Listed bellow is the work that needed to be done to satisfy the requirements.

    \begin{itemize}
        \item Build a kernel module that implement a scalable interface for adding and removing probes from user space. The kernel module also needed to implement kamprobes event handlers that call eBPF programs.
        \item Build a user space library for interfacing with the kernel module.
    \end{itemize}

    As part of ensuring that the implementation is correct and usable, and evaluating the project I committed to:
    \begin{itemize}
        \item Build end-to-end test that demonstrate the implementation is working as intended.
        \item Write a demo application using my implementation of tracing to measure queue latency in io\_uring, an asynchronous IO system of the Linux kernel.
        \item Compare the performance of my implementation to existing implementations in the Linux kernel.
    \end{itemize}

\section{Background} 
    The following section will describe material I needed to understand before beginning work on the project.

    \subsection{Instrumenting function calls - kamprobes}
        Since I relied on kamprobes for function call tracing, and I had to modify their implementation, I needed to learn how kamprobes work. C function calls are implemented using the call machine code instruction. Kamprobes change the program machine code while its running to change its behaviour, in particular to record the execution of a call instruction.

        A call instruction only has a single argument -- the address of the function being called, also known as the call destination. The x86-64 machine code representation of a call instruction would consist of a single byte operation code, and the 4 byte offset to the destination address (that is by how much we need to jump forwards, or backwards to get to the function address).

        If we want to trace a particular call to a function `foo(int a, int b)` we can do it by writing a new function, called for example trace\_foo(int a, int b), which stores the argument a and b into a buffer, calls the original function foo(a,b) and returns. We would then replace the 4 bytes of the function call argument to store the offset to trace\_foo instead of foo. When the call instruction is executed, trace\_foo is called instead of foo, and the function arguments are recorded. Because trace\_foo itself calls foo, apart from the arguments being recorded the effect of the call is the same.

        This is in principle how kamprobes work. Function trace\_foo is the event handler of the call instruction is executed event. Note that in the scheme I just described the event handler has to call the original function itself. Apart from being poor design because the handler might not call the original instruction, this design is problematic because you cannot simply use the same handler for multiple different functions. So what kamprobes really do is write machine code to memory that calls the handler and then the original function. They use the address of this generated machine code as the target of the traced call instruction.

    \subsection{Loadable kernel modules}
        The Linux kernel supports loading code that   

    \subsection{The eBPF virtual machine}
        System administrators often configure behaviour of the operating system. Examples of this are specifying which packets to filter, security policies, and also tracing event handlers. 

        Instead of adding more and more configuration options to define precise policies, the Linux kernel developers added a virtual machine, called eBPF, to the kernel. In this way policies can be configured in the form of programs for the eBPF virtual machine. Benefits of this approach are that eBPF programs are guaranteed to terminate as they are statically verifiend not to contain any loops. Further more eBPF programs may not write to kernel memory and the read they make are checked at runtime to be safe.

        A eBPF program is developed in C and compiled by the llvm compiler into bytecode. The bytecode is loaded into the kernel with the \texttt{bpf} systemcall. As a result of loading the program the user gets a file descriptor which they can use as a handle for any subsequent operations with the program. One such operation is attaching it to a probe.

    \subsection{The x86-64 calling convention}


\section{Choice of tools and languages}


\section{Development strategy}

\section{Starting point}

\section{Open source licensing}

\section{Summary}

\chapter{implementation}

    \section{Repository overview}

\printbibliography

\chapter*{Project Proposal}

%\centerline{\Large Computer Science Tripos -- Part II -- Project Proposal}
\vspace{6mm}
\centerline{\Large Optimizing tracing instrumentation inside the Linux kernel}
\vspace{6mm}
\centerline{\large D. Živanović, Trinity College}
\vspace{4mm}
\centerline{\large Originator: Dr Lucian Carata and Dr Ripduman Sohan}
\vspace{4mm}
\centerline{\large \today}

\vspace{8mm}

\noindent
{\bf Project Supervisor:} Dr Lucian Carata
\vspace{2mm}

\noindent
{\bf Director of Studies:} Prof Frank Stajano, Dr Sean Holden
\vspace{2mm}
 
\noindent
{\bf Project Overseers:} Dr Sean Holden, Dr Andreas Vlachos

%\vspace{4mm}

% Main document

% note function calls vs function call sites
\section*{Introduction}

    Dynamic tracing is an irreplaceable tool for analysing and debugging production systems. Tracing can be added to a 
    program without recompiling it or even restarting it. It has no overhead when disabled and adding a tracing
    probe changes only a few bytes of the program's instructions.
    By instrumenting the operating system's functions we can easily gain insight into how programs interact 
    with each other and use resources.

    %However, to be used in production the tracing mechanism must only minimally impact the performance of the 
    %traced program. Unfortunately this isn't the case with kprobes 

    Filtering out events, and summarizing them with a statistic can greatly reduce the amount of recorded data
    while tracing. In the Linux kernel this is done by specifying a user-supplied program to process function call events. 
    To guarantee safety and responsiveness this program is written in a restricted, non-Turing-complete 
    Instruction Set Architecture called eBPF. eBPF programs are guaranteed to halt in a limited number of steps,
    and to only perform valid memory accesses.

    The kprobes mechanism of the Linux kernel implements tracing by replacing the first byte of the 
    traced function with a software interrupt instruction whose handler does a hash table lookup on the 
    previous value of the instruction pointer to determine which callback function to call.
    This leads to some limitations, namely:

    \vspace{-0.60em}
    \begin{itemize}
        \setlength{\itemsep}{-0.3em}
        \item Poor scaling with the number of traced functions. This is due to the hash table used being limited in size.
        \item Inability to attach probes to specific call sites of a function, instead of function itself. 
        This is important when we want to trace calls to a function from a specific part of the kernel,
        and tracing all calls just to discard most of the data is prohibitively inefficient
    \end{itemize}

    Kamprobes, which are a probing mechanism developed by the Computer Lab's Digital Technology group, can instrument 
    specific call sites, but they don't support attaching eBPF programs as callbacks.

    The goal of this project is to enhance the tracing capabilities of the Linux kernel by allowing attaching eBPF programs
    to multiple call sites of a function in a low-overhead way. The plan is to do this by modifying the implementation of
    kamprobes and the Linux kernel to make it possible to register eBPF programs as kamprobes callbacks.
    I expect this enhancement to greatly reduce overhead of using eBPF with many probes, and when only some call sites
    should be traced.

    Furthermore I will be comparing the overhead of using eBPF programs with kamprobes and with kprobes. 
    Finally I will demonstrate the usefulness of eBPF and kamprobes by measuring some aspect of the Linux kernels
    performance.
 
\section*{Starting Point}

    This project will use the kamprobes kernel module as a dependency.
    The kamprobes module allows attaching callbacks functions to call instructions.
    Is currently does not have any interface accessible from the user space, and does not support
    attaching eBPF programs as callbacks.
    At some point extending the kamprobes module itself will likely prove convenient or necessary.

    The project will also depend on the Linux kernel, importantly, to run eBPF programs.
    I might need to modify parts of the kernel.

    %Since kamprobes is an open source project, this will be done in terms of an open source contribution.
    %In any case only the diff of the contribution will be considered as my contribution.

    I have previously written a basic, example Linux kernel module to familiarise myself with developing for the 
    Linux kernel. None of the code from that module will be used. 

\section*{Work to be done}

    The work to be done can be split into the following parts:

    \begin{itemize}
        \item Investigating how kprobes run eBPF programs, especially how the function arguments and the 
        current call stack is passed to the eBPF program.
       
        \item Getting familiar with the kamprobes project. Writing a simple callback function that logs 
        function arguments and attaching it to a call site in the kernel.

        \item Using the knowledge gained in the previous two steps to write the kamprobe callback that runs
        the eBPF program. In the most optimistic case this could be completely done inside a kernel module,
        but it may turn out that modifying the kernel source code is needed.

        \item Developing a user-space library for attaching an eBPF program to a kamprobe. This will involve 
        choosing a suitable user space to kernel interface (ioctl, device files, sys filesystem),
        and implementing the kernel module and library functionality.

        \item Developing test programs that shows that executing the traced call instruction results in calling the 
        eBPF program, and that the program can access the function arguments

        %\item Developing a test that ensures that executing the traced call instruction results in calling the 
        %eBPF program with the correct function arguments and call stack, that the function call is then executed
        %with the arguments and stack being unchanged, and that the instruction after the traced call is executed.

        %\item Developing a test that ensures that the implementation is thread safe, namely that concurrent requests
        %are handled correctly, and that probes firing concurrently don't cause any problems.

        \item Comparing the performance hit that an application suffers when kernel functions are traced with kprobes
        and with kamprobes. For example, the application can be nginx, and the performance hit can be measured
        in decrease of throughput and increase in latency. The measurements should show how performance changes when
        more probes are added.

        \item Demonstrating the usefulness of the implementation by measuring something about the kernel, for example
        the amount of time threads are blocked on some function.
    \end{itemize}

\section*{Resources required}

    I will do the project on my personal laptop (i5-8250U 1.60GHz quad-core CPU, 8GB of RAM, 256GB SSD, Ubuntu 18.04 LTS).
    
    For backup, all files will regularly be pushed to Github. Additionally, I will backup to an external hard disk
    once per week.

    Testing kernel modules, testing custom kernel builds if needed, and measuring performance will be done on 
    a Computer Lab owned server provided by my supervisor. To access the server {\bf I will need a Computer Lab account}.

\section*{Success criteria}

    This project is to be considered successful if:
    \begin{itemize}
        \item The kernel and user-space library code to instrument a function call with an eBPF program are implemented.
        Test programs demonstrating that function call arguments can be accessed by the eBPF program work.

        \item The usefulness of the project is demonstrated by measuring something about the kernel using 
        kamprobes and eBPF programs.

        \item The performance hit that an application suffers when instrumented with kprobes and kamprobes is 
        compared.

    \end{itemize}

\section*{Possible extensions}

    Possible extension to the project include:
    \begin{itemize}
        \item Adding a user friendly way to specify which call sites to instrument. Perhaps by specifying a
        source function, filename or module in which the function call is made, and the called function.

        \item Performing micro-architectural evaluation of the probing mechanisms using the performance counters
        provided by the CPU. These include counters of executed instructions, memory accesses, cache misses, and others.
        
        \item Adding specialised data structures to eBPF. The only memory eBPF programs can use that persists after an
        eBPF program has exited are eBPF maps. An eBPF map has one of a dozen possible types which are variations
        on arrays and hash maps. This extension deals with adding a more specialised eBPF maps to improve performance
        of eBPF tracing programs.
    \end{itemize}

\section*{Timetable, milestones and deadlines}

    Planned starting date is 24/10/2019. The work is split into 10 work packages:
    \begin{enumerate}

        \item {\bf Michaelmas weeks 3-4 (24/10/2019 - 6/11/2019)} 

            {\bf Deadline 25/10/2019:} Submission of the project proposal (this document) 

            Familiarise myself with the kamprobes kernel module. Compile and run the module.
            Attach a simple callback to a call site.

            Examine the kernel implementation of kprobes.

            Milestones: I know how to compile and use the kamprobes module.
            I know how kprobes run eBPF programs and how they pass function arguments and the call stack to the program.
        
        \item {\bf Michaelmas weeks 5-6 (07/11/2019 - 20/11/2019)} 

            Decide how to implement a kamprobes callback that runs eBPF programs.

            Start implementing the callback.

            Milestones: An implementation plan is made, and implementation has begun.

        \item {\bf Michaelmas weeks 7-8 (21/11/2019 - 04/12/2019)}

            Finish the callback implementation.

            Milestones: An eBPF program can be attached to a call site from a kernel module.

        \item {\bf Michaelmas vacation (05/12/2019 - 15/01/2020)}
        
            Implement the user-space library.
            Use the user-space library to write test programs.
            
            Write the introduction and preparation chapters of the dissertation.
            Send these chapters to the supervisor for review.

            Milestones: An eBPF program can be attached to a call site from user space.
            There is an automated test showing that the code works.
            A draft of implementation and preparation chapters is completed and sent for review.

        \item {\bf Lent weeks 1-2 (16/01/2020 - 29/01/2020)} 

            Implement performance tests.

            Prepare the progress report and presentation. Rehearse the presentation.

            Milestones: I have collected the performance data.
            The progress report and presentation are finished and the presentation is rehearsed.

        \item {\bf Lent weeks 3-4 (30/01/2020 - 12/02/2020)} 

            {\bf Deadline 31/01/2020:} Submission of the progress report 
            
            Choose a thing in the kernel to measure using kamprobes and eBPF,
            and write the eBPF programs that measure them.

            Milestones: The eBPF programs are written, and the metrics are measured using kamprobes.

        \item {\bf Lent weeks 5-6 (13/02/2020 - 26/02/2020)} 

            Either implement extensions or complete delayed work.

            Milestones: Work on extensions has begun.

        \item {\bf Lent weeks 7-8 (27/02/2020 - 11/03/2020)}

            Either implement extensions or complete delayed work.
            
            Begin working on implementation, and evaluation chapters.

            Milestones: All coding and evaluation including extensions has been completed. An early draft
            of the implementation and evaluation chapters is completed.

        \item {\bf Lent Vacation (12/03/2020 - 22/04/2020)}

            Finish the implementation, evaluation and conclusion chapters of the dissertation.

            {\bf Deadline 12/04/2020:} Send the first draft to my supervisor

            Incorporate feedback from my supervisor.

            {\bf Deadline 22/04/2020:} Send the final draft to my supervisor

            Milestones: The dissertation is complete and has already gone through one revision.

        \item {\bf Easter week 1(23/04/2020 - 29/04/2020)} 

            Polish the dissertation. Incorporate feedback from my supervisor.

            {\bf Deadline 01/05/2020:} Goal date for dissertation submission

            Milestones: The dissertation is submitted and the project is over.

    \end{enumerate}
    {\bf Deadline 08/05/2020:} Submission of the final dissertation


\end{document}
