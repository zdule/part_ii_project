% The master copy of this demo dissertation is held on my filespace
% on the cl file serve (/homes/mr/teaching/demodissert/)

% Last updated by MR on 2 August 2001

\documentclass[12pt,twoside,notitlepage]{report}

\usepackage{a4}
\usepackage{verbatim}
%\usepackage{navigator}
\usepackage{hyperref}
\hypersetup{
    colorlinks,
    citecolor={black},
    filecolor={black},
    linkcolor={black},
    urlcolor={black}
}

%\input{epsf}                            % to allow postscript inclusions
% On thor and CUS read top of file:
%     /opt/TeX/lib/texmf/tex/dvips/epsf.sty
% On CL machines read:
%     /usr/lib/tex/macros/dvips/epsf.tex



\raggedbottom                           % try to avoid widows and orphans
\sloppy
\clubpenalty1000%
\widowpenalty1000%

\addtolength{\oddsidemargin}{6mm}       % adjust margins
\addtolength{\evensidemargin}{-8mm}

\renewcommand{\baselinestretch}{1.1}    % adjust line spacing to make
                                        % more readable
\usepackage{biblatex}
\addbibresource{refs.bib}
\begin{document}
%%%%%%%%%%%%%%%%%%%%%%%%%%%%%%%%%%%%%%%%%%%%%%%%%%%%%%%%%%%%%%%%%%%%%%%%
% Title


\pagestyle{empty}

\hfill{\LARGE \bf Dušan Živanović}

\vspace*{60mm}
\begin{center}
\Huge
{\bf Optimizing tracing instrumentation inside the Linux kernel} \\
\vspace*{5mm}
Computer Science Tripos -- Part II \\
\vspace*{5mm}
Tirinty College \\
\vspace*{5mm}
2020
\end{center}


%%%%%%%%%%%%%%%%%%%%%%%%%%%%%%%%%%%%%%%%%%%%%%%%%%%%%%%%%%%%%%%%%%%%%%%%%%%%%%
% Proforma, table of contents and list of figures

\setcounter{page}{1}
\pagenumbering{roman}
\pagestyle{plain}

\chapter*{Acknowledgements}
\chapter*{Declaration}
    I, Dušan Živanović of Trinity College, being a candidate for Part II of the Computer Science Tripos, hereby declare that this dissertation and the work described in it are my own work, unaided except as may be specified below, and that the dissertation does not contain material that has already been used to any substantial extent for a comparable purpose. 

    I, Dušan Živanović of Trinity College, am content for my dissertation to be made available to the students and staff of the University. 

    Signed 

    Date 

    \bigskip
    \leftline{Signed [signature]}

    \medskip
    \leftline{Date [date]}


\chapter*{Proforma}


{\large
\begin{tabular}{ll}
Candidate Number:   & \bf TBD                   \\
Project Title:      & \bf Optimizing tracing instrumentation \\
                    & \bf inside the Linux kernel \\
Examination:        & \bf Computer Science Tripos -- Part II, 2020      \\
Word Count:         & \bf TBD \\
Line Count:         & \bf TBD \\
Project Originator: & Dr Lucian Carata and Dr Ripduman Sohan     \\
Supervisor:         & Dr Lucian Carata                  \\ 
\end{tabular}
}


\section*{Original Aims of the Project}


\section*{Work Completed}


\section*{Special Difficulties}



\tableofcontents

\listoffigures

%%%%%%%%%%%%%%%%%%%%%%%%%%%%%%%%%%%%%%%%%%%%%%%%%%%%%%%%%%%%%%%%%%%%%%%
% now for the chapters

\cleardoublepage        % just to make sure before the page numbering
                        % is changed

\setcounter{page}{1}
\pagenumbering{arabic}
\pagestyle{headings}

\chapter{Introduction}
%\outline{0}{Introduction}
    This dissertation introduces kambpf, a flexible, low-overhead function-call tracing system for the Linux kernel. Compared to the tracing capabilities currently in the Linux kernel, kambpf has a lower overhead and the added flexibility instrumenting any subset of call sites of a given function. Compared to the kamprobes project on which it builds upon, kambpf provides the flexibility of being programmable with eBPF, and an interface from the user space. 

    The kambpf system is available as a loadable kernel module with its functionality exported to the user space via a simple to use C library.

    \section{Motivation}
        %There is a class of bugs and performance issues that are impractical or impossible to be reproduced by a developer in a testing environment
        The behaviour and performance of a computer system is influenced by the software components it runs, the hardware on which it runs, how the system is configured, and the data it processes. It is thus inevitable that problems which were not be observed, and are difficult to reproduce in development and testing occur in production. In such cases it is often up to the system integrator to analyse the performance issues.\cite{DTrace2004}
        
        Instrumenting operating system's kernel, because it is the interface through which all processes interact with each other and with the hardware, is a great way to observe and analyse the behaviour of all processes running on the system. By instrumenting the kernel the system integrator can find performance bottlenecks and processes that are misbehaving. In particular, dynamic tracing is a great way to instrument long-running software such as the kernel, because it doesn't require recompiling or restarting to enable and disable the instrumentation.\cite{DynamicPI1994}

    \section{Previous work}

\chapter{Preparation}
%\outline{0}{Preparation}
    \section{Theoretical background} 
        \subsection{Linux kernel loadable modules}
        \subsection{kprobes}
        \subsection{kamprobes}
        \subsection{eBPF}
   
    \section {Requirements analysis} 
        
    \section{Choice of tools and languages}

        - scarce documentation, so reading source

    \section{Development strategy}

    \section{Starting point}

    \section{Open source licensing}

    \section{Summary}

\printbibliography

\chapter*{Project Proposal}

%\centerline{\Large Computer Science Tripos -- Part II -- Project Proposal}
\vspace{6mm}
\centerline{\Large Optimizing tracing instrumentation inside the Linux kernel}
\vspace{6mm}
\centerline{\large D. Živanović, Trinity College}
\vspace{4mm}
\centerline{\large Originators: Dr Lucian Carata and Dr Ripduman Sohan}
\vspace{4mm}
\centerline{\large \today}

\vspace{8mm}

\noindent
{\bf Project Supervisor:} Dr Lucian Carata
\vspace{2mm}

\noindent
{\bf Director of Studies:} Prof Frank Stajano, Dr Sean Holden
\vspace{2mm}
 
\noindent
{\bf Project Overseers:} Dr Sean Holden, Dr Andreas Vlachos

%\vspace{4mm}

% Main document

\section*{Introduction}

    Dynamic tracing is an irreplaceable tool for analysing and debugging production systems. 
    Some of the reasons tracing is so useful are that tracing has no overhead 
    when disabled and that it can be added to a program without recompiling it or even restarting it. 
    Furthermore, by instrumenting the kernel of an operating system we can easily gain insight into
    the workings of any program running on the machine.

    To be used in production a tracing mechanism mustn't significantly impact the performance of the 
    traced program. Unfortunately, tracing in the Linux kernel can scale poorly with increasing the number of active probes.
    The goal of this project is to enable low-overhead tracing with many active probes for the purpose of
    understanding the system's behavior in detail.

    As a first step towards enhancing the tracing in Linux we should examine the mechanisms and capabilities
    which are currently provided, namely eBPF programs and kprobes.

    The job of an eBPF program is to process the traced events as they occur.
    eBPF programs are important because preprocessing the measurement data before it is recorded to 
    memory can drastically reduce the tracing overhead. 
    For example if we are only interested in an average of some measurements it is much better to keep 
    a running average using eBPF while collecting the data, than to record all the data about events and 
    then process it later.

    eBPF programs are often one-off programs run inside the kernel and written to diagnose an urgent problem. 
    As such there is a risk that an erroneous program could crash the kernel or make it unresponsive.
    To ensure this never happens, eBPF programs are written in an restricted non-Turing-complete 
    Instruction Set Architecture and run by a virtual machine inside the kernel.

    One of the tracing mechanisms to which eBPF programs can be attached is kprobes. 
    Kprobes implement tracing by replacing the first byte of the traced function with a software interrupt 
    instruction whose handler does a hash table lookup on the 
    previous value of the instruction pointer to determine which callback function to invoke.
    This leads to some limitations, namely:

    \vspace{-0.60em}
    \begin{itemize}
        \setlength{\itemsep}{-0.3em}
        \item Poor scaling with the number of active probes. This is because the hash table used is limited in size.
        \item Inability to attach probes to specific call sites of a function, instead of function itself. 
        This is important when we want to trace calls to a function from a specific part of the kernel,
        and tracing all calls just to discard most of the data is prohibitively inefficient
    \end{itemize}

    Kamprobes, which are an alternative probing mechanism developed by the Computer Lab's Digital Technology group, 
    can instrument specific call sites and scale better with increasing the number of active probes, 
    but they don't support attaching eBPF programs as callbacks, and have limitations in how function arguments
    are accessed. Additionally, kamprobes don't have any interfaces to the user space, 
    and require the user to know the kernel space addresses of call sites to be traced,
    which is information not readily available outside the kernel.

    This project aims to enhance the tracing capabilities of the Linux kernel by allowing attaching eBPF programs
    to multiple call sites of a function in a scalable, low-overhead way. 
    The plan is to do this by modifying the implementation of
    kamprobes and the Linux kernel to make it possible to register eBPF programs as kamprobes callbacks.
    I expect this enhancement to greatly reduce overhead of using eBPF with many probes, and when only some call sites
    should be traced.

    Furthermore I will be comparing the overhead of using eBPF programs with kamprobes and with kprobes. 
    Finally I will demonstrate the usefulness of eBPF and kamprobes by measuring detailed information 
    about waiting in a subsystem of the Linux kernel.
 
\section*{Starting Point}

    This project will use the kamprobes kernel module as a dependency.
    The kamprobes module allows attaching callbacks functions to call instructions.
    Is currently does not have any interface accessible from the user space, and does not support
    attaching eBPF programs as callbacks.
    At some point extending the kamprobes module itself will likely prove convenient or necessary.

    The project will also depend on the Linux kernel, importantly, to run eBPF programs.
    I might need to modify parts of the kernel.

    %Since kamprobes is an open source project, this will be done in terms of an open source contribution.
    %In any case only the diff of the contribution will be considered as my contribution.

    I have previously written a basic, example Linux kernel module to familiarise myself with developing for the 
    Linux kernel. None of the code from that module will be used. 

\section*{Work to be done}

    The work to be done can be split into the following parts:

    \begin{itemize}
        \item Investigating how eBPF programs run on top of kamprobes, especially how the program
        accesses the function arguments and the current call stack.
       
        \item Getting familiar with the kamprobes project. Writing a simple callback function that logs 
        function arguments and attaching it to a call site in the kernel.

        \item Using the knowledge gained in the previous two steps to write the kamprobe callback that runs
        the eBPF program. In the most optimistic case this could be completely done inside a kernel module,
        but it may turn out that modifying the kernel source code is needed.

        \item Developing a user-space library for attaching an eBPF programs to a kamprobes. This will involve 
        choosing a suitable user space to kernel interface (ioctl, device files, sys filesystem),
        and implementing the kernel module and library functionality.

        \item Developing test programs that shows that executing the traced call instruction results in calling the 
        eBPF program, and that the program can access the function arguments.

        %\item Developing a test that ensures that executing the traced call instruction results in calling the 
        %eBPF program with the correct function arguments and call stack, that the function call is then executed
        %with the arguments and stack being unchanged, and that the instruction after the traced call is executed.

        %\item Developing a test that ensures that the implementation is thread safe, namely that concurrent requests
        %are handled correctly, and that probes firing concurrently don't cause any problems.

        \item Comparing the performance hit that an application suffers when kernel functions are traced with kprobes
        and with kamprobes. 
        For example, I could measure the decrease in throughput and increase in latency that nginx web server suffers
        when parts of the kernel's network stack are traced.
        Measurements should also be done with an increasing number of active probes to show differences in scaling
        between kprobes and kamprobes.

        \item Demonstrating the usefulness of the implementation by measuring data related to waiting in a subsytstem
        of the kernel.
    \end{itemize}


\section*{Success criteria}

    This project is to be considered successful if:
    \begin{itemize}
        \item The kernel and user-space library code to instrument a function call with an eBPF program are implemented.
        Test programs demonstrating that function call arguments can be accessed by the eBPF program work.

        \item The usefulness of the project is demonstrated by measuring detailed information about waiting in 
        a kernel subsystem using kamprobes and eBPF programs.

        \item The performance hit that an application suffers when instrumented with kprobes and kamprobes is 
        compared.

    \end{itemize}

\section*{Possible extensions}

    Possible extension to the project include:
    \begin{itemize}
        \item Adding a user friendly way to specify which call sites to instrument. Perhaps by specifying a
        source function, filename or module in which the function call is made, and the called function.

        \item Performing micro-architectural evaluation of the probing mechanisms using the performance counters
        provided by the CPU. These include counters of executed instructions, memory accesses, cache misses, and others.
        
        \item Adding specialised data structures to eBPF. The only memory eBPF programs can use that persists after an
        eBPF program has exited are eBPF maps. An eBPF map has one of a dozen possible types which are variations
        on arrays and hash maps. This extension deals with adding a more specialised eBPF maps to improve performance
        of eBPF tracing programs.
    \end{itemize}

\section*{Timetable, milestones and deadlines}

    Planned starting date is 24/10/2019. The work is split into 10 work packages:
    \begin{enumerate}

        \item {\bf Michaelmas weeks 3-4 (24/10/2019 - 6/11/2019)} 

            {\bf Deadline 25/10/2019:} Submission of the project proposal (this document) 

            Familiarise myself with the kamprobes kernel module. Compile and run the module.
            Attach a simple callback to a call site.

            Examine the kernel implementation of kprobes.

            Milestones: I know how to compile and use the kamprobes module.
            I know how kprobes run eBPF programs and how they pass function arguments and the call stack to the program.
        
        \item {\bf Michaelmas weeks 5-6 (07/11/2019 - 20/11/2019)} 

            Decide how to implement a kamprobes callback that runs eBPF programs.

            Start implementing the callback.

            Milestones: An implementation plan is made, and implementation has begun.

        \item {\bf Michaelmas weeks 7-8 (21/11/2019 - 04/12/2019)}

            Finish the callback implementation.

            Milestones: An eBPF program can be attached to a call site from a kernel module.

        \item {\bf Michaelmas vacation (05/12/2019 - 15/01/2020)}
        
            Implement the user-space library.
            Use the user-space library to write test programs.
            
            Write the introduction and preparation chapters of the dissertation.
            Send these chapters to the supervisor for review.

            Milestones: An eBPF program can be attached to a call site from user space.
            There is an automated test showing that the code works.
            A draft of implementation and preparation chapters is completed and sent for review.

        \item {\bf Lent weeks 1-2 (16/01/2020 - 29/01/2020)} 

            Implement performance tests.

            Prepare the progress report and presentation. Rehearse the presentation.

            Milestones: I have collected the performance data.
            The progress report and presentation are finished and the presentation is rehearsed.

        \item {\bf Lent weeks 3-4 (30/01/2020 - 12/02/2020)} 

            {\bf Deadline 31/01/2020:} Submission of the progress report 
            
            Measure metrics about waiting inside a module of the Linux kernel using kamprobes and eBPF.

            Milestones: The eBPF programs are written, and the metrics are measured using kamprobes.

        \item {\bf Lent weeks 5-6 (13/02/2020 - 26/02/2020)} 

            Either implement extensions or complete delayed work.

            Milestones: Work on extensions has begun.

        \item {\bf Lent weeks 7-8 (27/02/2020 - 11/03/2020)}

            Either implement extensions or complete delayed work.
            
            Begin working on implementation, and evaluation chapters.

            Milestones: All coding and evaluation including extensions has been completed. An early draft
            of the implementation and evaluation chapters is completed.

        \item {\bf Lent Vacation (12/03/2020 - 22/04/2020)}

            Finish the implementation, evaluation and conclusion chapters of the dissertation.

            {\bf Deadline 12/04/2020:} Send the first draft to my supervisor

            Incorporate feedback from my supervisor.

            {\bf Deadline 22/04/2020:} Send the final draft to my supervisor

            Milestones: The dissertation is complete and has already gone through one revision.

        \item {\bf Easter week 1(23/04/2020 - 29/04/2020)} 

            Polish the dissertation. Incorporate feedback from my supervisor.

            {\bf Deadline 01/05/2020:} Goal date for dissertation submission

            Milestones: The dissertation is submitted and the project is over.

    \end{enumerate}
    {\bf Deadline 08/05/2020:} Submission of the final dissertation

\section*{Resources required}

    I will do the project on my personal laptop (i5-8250U 1.60GHz quad-core CPU, 8GB of RAM, 256GB SSD, Ubuntu 18.04 LTS).
    
    For backup, all files will regularly be pushed to Github. Additionally, I will backup to an external hard disk
    once per week.

    Testing kernel modules, testing custom kernel builds if needed, and measuring performance will be done on 
    a Computer Lab owned server provided by my supervisor. To access the server {\bf I will need a Computer Lab account}.

\end{document}
